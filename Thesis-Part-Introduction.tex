\chapter{Introduction}



In addition to circumventing some measurement related paradoxes, our description of quantum mechanics contains other tangible advantages. We show that when simulating sequential Stern-Gerlach experiments, the program control flow becomes drastically simplified. This is a consequence of incorporating the branching structure of quantum mechanics directly into the data structure representing the quantum system, which follows naturally from the objects resulting from von Neumann measurement. Without this, the branching structure manifests through recursive loops in the program, adding unnecessary complexity.

Descriptions of measuring spin-$\frac{1}{2}$ systems in the consistent histories approach are exemplified in works by Griffiths and Hohenberg (TODO: cite). However, they either require prior knowledge of concepts in quantum foundations, or neglect implementing von Neumann measurement. We describe these concepts while developing our model of the Stern-Gerlach experiment, and show that the role of the apparatus described in von Neumann measurement must be included for a consistent description of the interaction. Additionally, von Nemann measurement is typically described using a map of states before and after the interaction (TODO: cite Schlosshauer); we introduce an explicit unitary operator that accomlishes this mapping.

In this thesis, we abandon the projection postulate, and instead adopt the \textit{von Neumann measurement scheme} to reproduce the measurement statistics of successive Stern-Gerlach experiments. The behavior described by this scheme is permitted by all other postulates, and resolves several paradoxes that plague quantum mechanics (TODO: ref to future sections). This scheme is central to the study of \textit{decoherence}, which describes the quantum to classical transition. The von Neumann description of measurement produces the same statistics without reference to the fifth postulate. Rather than postulating state collapse, we describe our experimental setup through an interseting Hamiltonian, through which the system evolves unitarily. Measurement is itself modeled as a physical process, rather than  postulated non-unitary dynamics.

Describing measurement in this way motivates the employment of an interpretation of quantum mechanics which does not require state collapse. The \textit{consistent histories} interpretation assigns physical meaning to the mathematical objects used to model quantum systems, and prescribes strict rules of reasoning to be used for them. The result is that a \textit{quantum history} becomes the object for which quantum theory makes predictions, and it can be used to elegantly describe the sequential events regarding the system. It is one of many interpretations employing the \textit{relative states} formalism, which makes von Neumann measurement central. Understanding how quantum states decay in time is accomplished by studying \textit{decoherence}, which is also expressed in terms of the von Neumann measurement scheme.


This is accomplished by modeling the measurement apparatus itself as a quantum system and redefining the measurement interaction. This extension of quantum mechanics is necessary for answering questions in cosmology, effectively allowing state collapse to occur in the early universe \cite{Craig}.

Numerous papers and books describe these concepts in detail, but they have yet to permeate far outside the quantum foundations community. The goal of this thesis, then, is to introduce these concepts in a more accessible form. Following the lead of research in physics education purporting the effectiveness of a spins-first introduction to quantum mechanics, we introduce the consistent histories approach using spin-$\frac{1}{2}$ systems. Having only two degrees of freedom, these are the simplest possible systems, and all fundamental aspects of quantum mechanics can explained in the context of the Stern-Gerlach experiment.

In conclusion, we examine the \textit{Wigner's friend} experiment in which the standard and von Neumann descriptions of measurement make different predictions. We discuss existing and future experiments that may distingush which formalism correctly represents physical reality.
