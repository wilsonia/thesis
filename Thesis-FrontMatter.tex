% Use Roman numerals (i, ii, iii, etc.) for page numbers in the front matter.
\pagenumbering{roman}

%%%%%%%%%%%%%%%%%%%%%%%%%%%%%%%%%%%%%%%%%%%%%%%%%%%%%%%%%%%%%%%%%
%% TITLE PAGE.
%%%%%%%%%%%%%%%%%%%%%%%%%%%%%%%%%%%%%%%%%%%%%%%%%%%%%%%%%%%%%%%%%

% No headers or footers on the title page.
\thispagestyle{empty}

\begingroup
\centering
\setstretch{1.0}
~
\\[1em]
\sffamily\bfseries\fontsize{26}{31.2}\selectfont
\DocumentTitle
\\[0.6in]
\sffamily\bfseries\Large
\AuthorName
\\[0.6in]
\normalfont\normalsize
An undergraduate thesis advised by Dr. David Craig \\~\\
submitted to the Department of Physics, Oregon State University \\~\\
in partial fulfillment of the requirements for the degree BSc in Physics \\[0.6in]
Submitted on May 8, 2020
\vfill
\includegraphics[height=1.0in]
{Figure-SchoolLogo}
\par
\endgroup

\clearpage

%%%%%%%%%%%%%%%%%%%%%%%%%%%%%%%%%%%%%%%%%%%%%%%%%%%%%%%%%%%%%%%%%
%% COPYRIGHT PAGE.
%%%%%%%%%%%%%%%%%%%%%%%%%%%%%%%%%%%%%%%%%%%%%%%%%%%%%%%%%%%%%%%%%

% \pagestyle{plain}
% \setcounter{page}{2}
%
% \begingroup
% \centering
% \setstretch{1.0}
% \null
% \vfill
% {\sffamily\textcopyright}~2016
% \\[0.5em]
% \AuthorName
% \\[0.5em]
% All Rights Reserved
% \par
% \endgroup
%
% \clearpage

% %%%%%%%%%%%%%%%%%%%%%%%%%%%%%%%%%%%%%%%%%%%%%%%%%%%%%%%%%%%%%%%%%
% %% DEDICATION PAGE.
% %%%%%%%%%%%%%%%%%%%%%%%%%%%%%%%%%%%%%%%%%%%%%%%%%%%%%%%%%%%%%%%%%
%
% \begingroup
% \centering
% \setstretch{1.0}
% ~
% \\[1in]
% \textit{Insert dedication here}
% \par
% \endgroup
%
% \clearpage

%%%%%%%%%%%%%%%%%%%%%%%%%%%%%%%%%%%%%%%%%%%%%%%%%%%%%%%%%%%%%%%%%
%% ACKNOWLEDGMENTS.
%%%%%%%%%%%%%%%%%%%%%%%%%%%%%%%%%%%%%%%%%%%%%%%%%%%%%%%%%%%%%%%%%

\chapter*{Acknowledgments}
\addcontentsline{toc}{chapter}{Acknowledgments}

I would like to thank Dr. David Craig, for guiding me through an engaging line of research, as well as Dr. David McIntyre, Dr. Elizabeth Gire, Dr. Corinne Manogue and Dr. Janet Tate for developing the quantum curriculum from which this thesis is rooted.

I would also like to thank all of those who gave me the time, space, and support I needed while writing this. This includes (but is certainly not limited to) my partner Brooke, my parents Joy and Kevin, my housemate Cheyanne, the staff of Interzone, and my friends Saskia, Rachel and Justin.

\clearpage

%%%%%%%%%%%%%%%%%%%%%%%%%%%%%%%%%%%%%%%%%%%%%%%%%%%%%%%%%%%%%%%%%
%% ABSTRACT.
%%%%%%%%%%%%%%%%%%%%%%%%%%%%%%%%%%%%%%%%%%%%%%%%%%%%%%%%%%%%%%%%%

\chapter*{Abstract}
\addcontentsline{toc}{chapter}{Abstract}

Standard quantum mechanics makes foundational assumptions to describe the measurement process. Upon interaction with a ``classical measurement apparatus'', a quantum system is subjected to postulated ``state collapse'' dynamics. We show that framing measurement around state collapse and ill-defined classical observers leads to interpretational issues, and artificially limits the scope of quantum theory. This motivates describing measurement as a unitary process instead. In the context of the Stern-Gerlach experiment, the measurement of an electron's spin angular momentum is explained as the entanglement of its spin and position degrees of freedom. Furthermore, the electron-environment interaction is also detailed as part of the measurement process. The environment plays the role of a record keeper, establishing the ``facts of the universe'' to make the measurement's occurrence objective. This enables the assignment of probabilities to outcomes of consecutive Stern-Gerlach experiments, which is done using the consistent histories approach to quantum mechanics. We exemplify how consistent histories can be used to give a unitary measurement model predictive power by reproducing the predictions of the standard approach. In doing so, we provide a ``spins-first'' introduction to the consistent histories approach that is targeted towards anybody looking to learn more about quantum foundations. Along the way, we coin a mathematically convenient operator that simplifies the calculation of probabilities. We conclude by discussing how consistent histories ignores the ``problem of outcomes'' in quantum measurement theory, and the connection between this problem and the many worlds interpretation of quantum mechanics.

% A novel approach to limiting extraneous predictions implements the principle of complementarity into the measurement process directly.

\clearpage

%%%%%%%%%%%%%%%%%%%%%%%%%%%%%%%%%%%%%%%%%%%%%%%%%%%%%%%%%%%%%%%%%
%% TABLE OF CONTENTS (TOC), LISTS OF FIGURES, TABLES, ETC.
%%%%%%%%%%%%%%%%%%%%%%%%%%%%%%%%%%%%%%%%%%%%%%%%%%%%%%%%%%%%%%%%%

\tableofcontents

\listoffigures

\clearpage

% Use Arabic numerals (1, 2, 3, etc.) for subsequent page numbers.
\pagenumbering{arabic}
