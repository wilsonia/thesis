% Use Roman numerals (i, ii, iii, etc.) for page numbers in the front matter.
\pagenumbering{roman}

%%%%%%%%%%%%%%%%%%%%%%%%%%%%%%%%%%%%%%%%%%%%%%%%%%%%%%%%%%%%%%%%%
%% TITLE PAGE.
%%%%%%%%%%%%%%%%%%%%%%%%%%%%%%%%%%%%%%%%%%%%%%%%%%%%%%%%%%%%%%%%%

% No headers or footers on the title page.
\thispagestyle{empty}

\begingroup
\centering
\setstretch{1.0}
~
\\[1em]
\sffamily\bfseries\fontsize{26}{31.2}\selectfont
\DocumentTitle
\\[0.6in]
\sffamily\bfseries\Large
\AuthorName
\\[0.6in]
\normalfont\normalsize
An undergraduate thesis advised by Dr. David Craig \\~\\
submitted to the Department of Physics, Oregon State University \\~\\
in partial fullfillment of the requirements for the degree BSc in Physics \\[0.6in]
Submitted on February 25, 2020
\vfill
\includegraphics[height=1.0in]
{Figure-SchoolLogo}
\par
\endgroup

\clearpage

%%%%%%%%%%%%%%%%%%%%%%%%%%%%%%%%%%%%%%%%%%%%%%%%%%%%%%%%%%%%%%%%%
%% COPYRIGHT PAGE.
%%%%%%%%%%%%%%%%%%%%%%%%%%%%%%%%%%%%%%%%%%%%%%%%%%%%%%%%%%%%%%%%%

% \pagestyle{plain}
% \setcounter{page}{2}
%
% \begingroup
% \centering
% \setstretch{1.0}
% \null
% \vfill
% {\sffamily\textcopyright}~2016
% \\[0.5em]
% \AuthorName
% \\[0.5em]
% All Rights Reserved
% \par
% \endgroup
%
% \clearpage

% %%%%%%%%%%%%%%%%%%%%%%%%%%%%%%%%%%%%%%%%%%%%%%%%%%%%%%%%%%%%%%%%%
% %% DEDICATION PAGE.
% %%%%%%%%%%%%%%%%%%%%%%%%%%%%%%%%%%%%%%%%%%%%%%%%%%%%%%%%%%%%%%%%%
%
% \begingroup
% \centering
% \setstretch{1.0}
% ~
% \\[1in]
% \textit{Insert dedication here}
% \par
% \endgroup
%
% \clearpage

%%%%%%%%%%%%%%%%%%%%%%%%%%%%%%%%%%%%%%%%%%%%%%%%%%%%%%%%%%%%%%%%%
%% ACKNOWLEDGMENTS.
%%%%%%%%%%%%%%%%%%%%%%%%%%%%%%%%%%%%%%%%%%%%%%%%%%%%%%%%%%%%%%%%%

\chapter*{Acknowledgments}
\addcontentsline{toc}{chapter}{Acknowledgments}

TODO: insert acknowledgments

\clearpage

%%%%%%%%%%%%%%%%%%%%%%%%%%%%%%%%%%%%%%%%%%%%%%%%%%%%%%%%%%%%%%%%%
%% ABSTRACT.
%%%%%%%%%%%%%%%%%%%%%%%%%%%%%%%%%%%%%%%%%%%%%%%%%%%%%%%%%%%%%%%%%

\chapter*{Abstract}
\addcontentsline{toc}{chapter}{Abstract}

Standard quantum mechanics makes foundational assumptions describing the measurement process. We show that postulating state collapse artifically limits the scope of quantum theory, motivating a unitary description of measurement. In the context of Stern-Gerlach experiments, we explain measurement as entanglement of the measured spin system, the measured position system, and the measuring apparatus. To interpret the entangled state, the consistent histories approach is used to make probabilistic predictions. To do this, the environment must also be entangled with the system, playing the role of a record keeper. The differing methods used by Griffiths and Gell-Mann/Hartle to incorporate the environment is articulated. We conclude by exemplifying tangible advantages of this approach by simplifying code that simulates consecutive Stern-Gerlach measurements.

% A novel approach to limiting extraneous predictions implements the principle of complementarity into the measurement process directly.

\clearpage

%%%%%%%%%%%%%%%%%%%%%%%%%%%%%%%%%%%%%%%%%%%%%%%%%%%%%%%%%%%%%%%%%
%% TABLE OF CONTENTS (TOC), LISTS OF FIGURES, TABLES, ETC.
%%%%%%%%%%%%%%%%%%%%%%%%%%%%%%%%%%%%%%%%%%%%%%%%%%%%%%%%%%%%%%%%%

\tableofcontents

\listoffigures

\clearpage

% Use Arabic numerals (1, 2, 3, etc.) for subsequent page numbers.
\pagenumbering{arabic}
