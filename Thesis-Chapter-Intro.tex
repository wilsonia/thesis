\usetikzlibrary{shapes.geometric}
\usetikzlibrary{positioning}

\newcommand{\apparatus}[4]{\node[square node] (#1) at (#2,#3){#4};
                           \node[port] (#1+) at (#2 + 0.375, #3 + 0.5){+};
                           \node[port] (#1-) at (#2 + 0.375, #3 - 0.5){-};}

\part{Background}
\chapter{Stern-Gerlach Experiments}

\section{Complementarity}

We now compare standard and consistent quantum mechanis' treatment of the principle of complementarity. Arguably the most fundamental feature of quantum mechanics, the principle of complementarity states that a quantum system has physical observables which cannot be measured simultaneously. Components of spin on orthogonal axes are complemntary properties, so we examine the measurements of succesive Stern-Gerlach experiments shown in \Figure~\fref{Figure:Intro:FigureExampleA}.

First, we compute the probabilities of each outcome using standard quantum mechanics. The first apparatus serves as a state preparation device with output $\ket{+}$. Measurement by the second apparatus (oriented along the $X$ axis) is decribed by the \textit{projection postulate} and the \textit{probability postulate}.

The projection postulate describes an instantaneous evolution of the input state to an output state that corresponds to an allowed measruement value. If input $\ket{\psi}$ is measured to have spin $n$ (up or down), then the new state is
\begin{align}
    \ket{\psi}^\prime = \frac{P_n\ket{\psi}}{\sqrt{\bra{\psi}P_n\ket{\psi}
    }}
\end{align}
where $P_n = \ket{n}\bra{n}$ is the projection operator for state $\ket{n}$. This operation projects $\ket{\psi}$ onto $\ket{n}$, then divides by the magnitude of that projection. The end result is that $\ket{\psi}$ becomes the normalized state $\ket{n}$ corresponding to measuring spin $n$, which is shown by $\ket{+}_x$ and $\ket{-}_x$ exiting the second apparatus.

\begin{figure}
\centering\CaptionFontSize
\begin{tikzpicture}[>=stealth',shorten >=1pt,auto, thick,
     square node/.style={rectangle, minimum height=2cm, minimum width=1.50cm, text width = 1cm, draw, font=\sffamily\Large\bfseries},
     port/.style={rectangle, draw,  minimum height=1cm, minimum width=0.75cm, font=\sffamily\Large\bfseries},
     wf/.style={rectangle, minimum height=1cm}]
    \apparatus{1}{3}{0}{Z};
    \apparatus{2}{6}{1}{X};
    \apparatus{3}{9}{1}{Z};

    \node[wf] (w0) at (0,0) {$\ket{\psi}$};
    \node[wf] (w1) at (12,2.25) {$\ket{+}$};
    \node[wf] (w2) at (12,-0.25) {$\ket{-}$};

    \draw[line width=0.5mm] (w0) -- (1);

    \draw[line width=0.5mm] (1+) -- (2) node [near end] {$\ket{+}$};

    \draw[line width=0.5mm] (2-) -- (3) node [midway, below] {$\ket{-}_x$};
    \draw[line width=0.5mm] (2+) -- (3) node [midway, above] {$\ket{+}_x$};

    \draw[line width=0.5mm] (3-) -- (w2);
    \draw[line width=0.5mm] (3+) -- (w1);
\end{tikzpicture}
\caption[Insert an abbreviated caption here to show in the List of Figures]
{Demonstrating interference of branch wavefunctions in Consistent Quantum Theory}
\label{Figure:Intro:FigureExampleA}
\end{figure}



\begin{figure}
\centering\CaptionFontSize
\begin{tikzpicture}[>=stealth',shorten >=1pt,auto, thick,
     square node/.style={rectangle, minimum height=2cm, minimum width=1.50cm, text width = 1cm, draw, font=\sffamily\Large\bfseries},
     port/.style={rectangle, draw,  minimum height=1cm, minimum width=0.75cm, font=\sffamily\Large\bfseries},
     wf/.style={rectangle, minimum height=1cm}]
    \apparatus{1}{3}{0}{Z};
    \apparatus{2}{6}{1}{X};
    \apparatus{3}{9}{1}{Z};

    \node[wf] (w0) at (0,0) {$\ket{\psi}$};
    \node[wf] (w1) at (12,2.5) {$\ket{\psi_1}$};
    \node[wf] (w2) at (12,1.5) {$\ket{\psi_2}$};
    \node[wf] (w3) at (12,0.5) {$\ket{\psi_3}$};
    \node[wf] (w4) at (12,-0.5) {$\ket{\psi_4}$};

    \draw[line width=0.5mm] (w0) -- (1);

    \draw[transform canvas={yshift=-0.6em}, line width=0.5mm, loosely dotted] (1+) -- (2);
    \draw[transform canvas={yshift=-0.2em}, line width=0.5mm, dotted] (1+) -- (2);
    \draw[transform canvas={yshift=0.2em}, line width=0.5mm, dashed] (1+) -- (2);
    \draw[transform canvas={yshift=0.6em}, line width=0.5mm] (1+) -- (2);

    \draw[transform canvas={yshift=-0.4em}, line width=0.5mm, loosely dotted] (2-) -- (3);
    \draw[transform canvas={yshift=0em}, line width=0.5mm, dotted] (2-) -- (3);
    \draw[transform canvas={yshift=0em}, line width=0.5mm, dashed] (2+) -- (3);
    \draw[transform canvas={yshift=0.4em}, line width=0.5mm] (2+) -- (3);

    \draw[line width=0.5mm, loosely dotted] (3-) -- (w4);
    \draw[line width=0.5mm, dashed] (3-) -- (w3);
    \draw[line width=0.5mm, dotted] (3+) -- (w2);
    \draw[line width=0.5mm] (3+) -- (w1);
\end{tikzpicture}
\caption[Insert an abbreviated caption here to show in the List of Figures]
{Demonstrating interference of branch wavefunctions in Consistent Quantum Theory}
\label{Figure:Intro:FigureExampleB}
\end{figure}
