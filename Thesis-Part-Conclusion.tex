% TODO: discuss any structural or algorithmic changes resulting from rewriting spins simulation
\chapter{Conclusion}
\bibliography{references}{}
\bibliographystyle{osu2}

% as we incorporate the branching structure of quantum mechanics directly into the data
% structure representing the quantum system, which follows naturally from the objects resulting from
% von Neumann measurement. Without this, the branching structure manifests through recursive
% loops in the program, adding unnecessary complexity.

\section{Consecutive Measurements}
TODO: introduce assumptions of consecutive measurement.

The von Neumann measurement scheme can be applied succesively.

\subsection{Example 2}
Now that we have two apparatuses, the Hilbert space includes two pointer spaces: $\mathcal{H} = \mathcal{H}_s \otimes \mathcal{H}_x \otimes \mathcal{H}_{a_1} \otimes \mathcal{H}_{a_2}$. We also add two new coarse grained position states for the second apparatus. The position states are now $\{\ket{+_{x_1}}, \ket{-_{x_1}}, \ket{+_{x_2}}, \ket{-_{x_2}}, \ket{\varnothing_x}\}$, where $\ket{\varnothing_x}$ is any position not in either apparatus' spin-up or spin-down region. TODO: figure.

The dynamics are unchanged from TODO REF Example 1 for the first measurement, with the identity acting on the second apparatus to leave it unaffected:
\begin{align}
  U(t_1, t_0)_{a_1} &= P^z_+ \otimes E^{x_1}_+  \nonumber \otimes E^{a_1}_+ \otimes I_{a_2}\\ \nonumber
  &+ P^z_- \otimes E^{x_1}_-\otimes E^{a_1}_- \otimes I_{a_2}
\end{align}

Now we determine the dynamics for the second measurement. We expect the dynamics to do two things: revers the entanglements from the first measurement, and entangle the second apparatus with spin and position eigenstates. We return the first apparatus to the ready state, as it is no longer measuring the system and no torque is exerted on the magnet, and return position back to $\ket{\varnothing_x}$ as the electron leaves the analyzer. $U(t_1, t_0)^\dagger$ reverses the entanglement, but conveniently, $U(t_1, t_0)$ is Hermitian. So, the $a_1$ component of the unitary operator for the second measurement is the same as that of the first measurement:
\begin{align}
  U(t_2, t_1)_{a_1} &= U(t_1, t_0)
\end{align}

For the $a_2$ component, we want to entangle pointer states with position and spin states only when the first apparatus has measured spin-up. In the branch where spin-down has been measured, we use the identity operator $I = I_s \otimes I_x \otimes I_{a_1} \otimes I_{a_2}$ to represent the lack of a second measurement. In the branch where spin-up has been measured, we apply the measurement scheme again. The $a_2$ component of the unitary operator during second measurement is
\begin{align}
  U(t_2, t_1)_{a_2} &= P^x_+ P^z_+ \otimes E^{x_2}_+ \otimes I_{a_1} \otimes E^{a_2}_+ \\ \nonumber
  &+ P^x_- P^z_+ \otimes E^{x_2}_- \otimes I_{a_1} \otimes E^{a_2}_- \\ \nonumber
  &+ P^z_- \otimes{I_x} \otimes I_{a_1} \otimes I_{a_2}
\end{align}

So the complete unitary operator for the second measurement is
\begin{align}
  U(t_2, t_1) &= U(t_2, t_1)_{a_2} U(t_2, t_1)_{a_1} \\ \nonumber
  &= P^x_+ P^z_+ \otimes E^{x_2}_+  E^{x_1}_+ \otimes E^{a_1}_+ \otimes E^{a_2}_+ \\ \nonumber
  &+ P^x_- P^z_+ \otimes E^{x_2}_-  E^{x_1}_+ \otimes E^{a_1}_+ \otimes E^{a_2}_- \\ \nonumber
  &+ P^z_- \otimes E^{x_1}_- \otimes E^{a_1}_- \otimes I_{a_2}
\end{align}
and the final state is represented in Figure TODO REF.
% \begin{align}
%   U(t_2, t_1) &= U(t_2, t_1)_{a_1} U(t_2, t_1)_{a_2} \\ \nonumber
%   &= P^x_+ P^z_+ \otimes \left(\ket{+_{a_1}}\bra{\varnothing_{a_1}} \: \bm{+} \: \ket{\varnothing_{a_1}}\bra{+_{a_1}} \: \bm{+} \: \ket{-_{a_1}}\bra{-_{a_1}} \right) \\ \nonumber
%   & \phantom{{} = P^x_+ P^z_+} \otimes  \left(\ket{+_{a_2}}\bra{\varnothing_{a_2}} \: \bm{+} \: \ket{\varnothing_{a_2}}\bra{+_{a_2}} \: \bm{+} \: \ket{-_{a_2}}\bra{-_{a_2}} \right) \\ \nonumber
%   &+ P^x_- P^z_+ \otimes \left(\ket{+_{a_1}}\bra{\varnothing_{a_1}} \: \bm{+} \: \ket{\varnothing_{a_1}}\bra{+_{a_1}} \: \bm{+} \: \ket{-_{a_1}}\bra{-_{a_1}} \right) \\ \nonumber
%   & \phantom{{} = P^x_+ P^z_+} \otimes  \left(\ket{-_{a_2}}\bra{\varnothing_{a_2}} \: \bm{+} \: \ket{\varnothing_{a_2}}\bra{-_{a_2}} \: \bm{+} \: \ket{+_{a_2}}\bra{+_{a_2}} \right) \\ \nonumber
%   &+ P^z_- \otimes \left(\ket{-_{a_1}}\bra{\varnothing_{a_1}} \: \bm{+} \: \ket{\varnothing_{a_1}}\bra{-_{a_1}} \: \bm{+} \: \ket{+_{a_1}}\bra{+_{a_1}} \right) \\ \nonumber
%   & \phantom{{}={}P^z_+} \otimes I_{a_2}
% \end{align}

\begin{figure}
\centering\CaptionFontSize

\begin{tikzpicture}[shorten >=1pt,auto, thick,
     square node/.style={rectangle, minimum height=2cm, minimum width=1.50cm, text width = 1.25cm, draw, font=\sffamily\Large\bfseries},
     port/.style={rectangle, draw,  minimum height=1cm, minimum width=0.75cm, font=\sffamily\Large\bfseries},
     wf/.style={rectangle, minimum height=1cm}]
    \apparatus{1}{4}{0}{${}^{a_1}_z$};
    \apparatus{2}{6.5}{1.5}{${}^{a_2}_x$};

    \node(w0) at (0.25,0) {$\ket{\psi_s} \otimes \ket{\varnothing_x} \otimes \ket{\varnothing_{a_1}} \otimes \ket{\varnothing_{a_2}} $};
    \node[wf] (w1) at (8, -.5) {$P^z_-\ket{\psi_s} \otimes \ket{-_{x_1}}  \otimes \ket{-_{a_1}} \otimes \ket{\varnothing_{a_2}}$};
    \node[wf] (w2) at (10.75, 1) {$P^x_-P^z_+\ket{\psi}_s \otimes \ket{-_{x_2}} \otimes \ket{\varnothing_{a_1}} \otimes \ket{-_{a_2}}$};
    \node[wf] (w3) at (10.75, 2) {$P^x_+ P^z_+\ket{\psi_s} \otimes \ket{+_{x_2}} \otimes \ket{\varnothing_{a_1}} \otimes \ket{+_{a_2}}$};

    % \node(label0) at (0, -1.75) {$\bm{t_0}$};
    % \node(label1) at (4.75, -1.75) {$\bm{t_1}$};
    % \node(label2) at (7.25, -1.75) {$\bm{t_2}$};
    % \node(label3) at (9.75, -1.75) {$\bm{t_3}$};
    % \node(bw1) at (-1.75, 1.75) {$P^z_+\ket{\psi}^s \otimes \ket{+}^{D_1}_z \otimes \ket{\varnothing}^{D_2}_x $};

    \draw[line width=0.5mm] (w0) -- (1);
    \draw[line width=0.5mm] (1+) -- (2);
    \draw[line width=0.5mm] (1-) --  (w1);
    \draw[line width=0.5mm] (2+) -- (w3);
    \draw[line width=0.5mm] (2-) -- (w2);


\end{tikzpicture}

\caption[Insert an abbreviated caption here to show in the List of Figures]
{TODO: caption}
\label{Figure:Measurement:labelthis2}
\end{figure}

% \subsection{Example 3}
%
%
% \begin{figure}
% \centering\CaptionFontSize
% % \begin{tikzpicture}[shorten >=1pt,auto, thick,
% %      square node/.style={rectangle, minimum height=2cm, minimum width=1.50cm, text width = 1.25cm, draw, font=\sffamily\Large\bfseries},
% %      port/.style={rectangle, draw,  minimum height=1cm, minimum width=0.75cm, font=\sffamily\Large\bfseries},
% %      wf/.style={rectangle, minimum height=1cm}]
% %     \apparatus{1}{-4}{0}{$Z$};
% %     \apparatus{2}{2}{1.5}{$X$};
% %     \apparatus{3}{2}{-1.5}{$X$};
% %
% %     \node(w0) at (-8,0) {$\ket{\psi}^s \otimes \ket{\varnothing}^{D_1}_z \otimes \ket{\varnothing}^{D_2}_x \otimes \ket{\varnothing}^{D_3}_x$};
% %     \node[wf] (w1) at (6, 2) {$P^x_+ P^z_+\ket{\psi}^s \otimes \ket{\varnothing}^{D_1}_z \otimes \ket{+}^{D_2}_x \otimes \ket{\varnothing}^{D_3}_x$};
% %     \node[wf] (w2) at (8.25, 1) {${P^x}_-{P^z}_-\ket{\psi}_s \otimes \ket{\mathcal{X}_-}_z$};
% %     \node[wf] (w3) at (8.25, -1) {${P^z}_+\ket{\psi}_s \otimes \ket{\mathcal{X}_+}_z$};
% %     \node[wf] (w4) at (8.25, -2) {${P^z}_-\ket{\psi}_s \otimes \ket{\mathcal{X}_-}_z$};
% %
% %     % \node(label1) at (0, -1.75) {$\bm{t_0}$};
% %     % \node(label2) at (6.25, -1.75) {$\bm{t_1}$};
% %     \node(bw1) at (-1.75, 1.75) {$P^z_+\ket{\psi}^s \otimes \ket{+}^{D_1}_z \otimes \ket{\varnothing}^{D_2}_x \otimes \ket{\varnothing}^{D_3}_x$};
% %
% %     \node(bw2) at (-1.75, -1.75) {$P^z_-\ket{\psi}^s \otimes \ket{-}^{D_1}_z \otimes \ket{\varnothing}^{D_2}_x \otimes \ket{\varnothing}^{D_3}_x$};
% %
% %     \draw[line width=0.5mm] (w0) -- (1);
% %     \draw[line width=0.5mm] (1+) -- (2);
% %     \draw[line width=0.5mm] (1-) --  (3);
% %     \draw[line width=0.5mm] (2+) -- (w1);
% %     \draw[line width=0.5mm] (2-) -- (w2);
% %     \draw[line width=0.5mm] (3+) -- (w3);
% %     \draw[line width=0.5mm] (3-) -- (w4);
% % \end{tikzpicture}
% \begin{adjustbox}{width=\textwidth}
% \begin{tikzpicture}[shorten >=1pt,auto, thick,
%      square node/.style={rectangle, minimum height=2cm, minimum width=1.50cm, text width = 1.25cm, draw, font=\sffamily\Large\bfseries},
%      port/.style={rectangle, draw,  minimum height=1cm, minimum width=0.75cm, font=\sffamily\Large\bfseries},
%      wf/.style={rectangle, minimum height=1cm}]
%     \apparatus{1}{4}{0}{${}^{D_1}_z$};
%     \apparatus{2}{6.5}{1.5}{${}^{D_2}_x$};
%     \apparatus{3}{9}{3}{${}^{D_3}_z$};
%
%     \node(w0) at (0,0) {$\ket{\psi_s} \otimes \ket{\varnothing_{D_1}} \otimes \ket{\varnothing_{D_2}} \otimes \ket{\varnothing_{D_3}}$};
%     \node[wf] (w1) at (8, -.5) {$ P^z_-\ket{\psi_s} \otimes \ket{-_{D_1}} \otimes \ket{\varnothing_{D_2}} \otimes \ket{\varnothing_{D_3}} $};
%     \node[wf] (w2) at (10.75, 1) {$P^x_-P^z_+\ket{\psi}_s \otimes \ket{\varnothing_{D_1}} \otimes \ket{-_{D_2}} \otimes \ket{\varnothing_{D_3}} $};
%     \node[wf] (w3) at (13.5, 3.5) {$P^z_+ P^x_+ P^z_+\ket{\psi_s} \otimes \ket{\varnothing_{D_1}} \otimes \ket{\varnothing_{D_2}} \otimes \ket{+_{D_3}}$};
%     \node[wf] (w4) at (13.5, 2.5) {$P^z_- P^x_+ P^z_+ \ket{\psi_s} \otimes \ket{\varnothing_{D_1}} \otimes \ket{\varnothing_{D_2}} \otimes \ket{-_{D_3}} $};
%
%     % \node(label0) at (0, -1.75) {$\bm{t_0}$};
%     % \node(label1) at (4.75, -1.75) {$\bm{t_1}$};
%     % \node(label2) at (7.25, -1.75) {$\bm{t_2}$};
%     % \node(label3) at (9.75, -1.75) {$\bm{t_3}$};
%     % \node(bw1) at (-1.75, 1.75) {$P^z_+\ket{\psi}^s \otimes \ket{+}^{D_1}_z \otimes \ket{\varnothing}^{D_2}_x $};
%
%     \draw[line width=0.5mm] (w0) -- (1);
%     \draw[line width=0.5mm] (1+) -- (2);
%     \draw[line width=0.5mm] (1-) --  (w1);
%     \draw[line width=0.5mm] (2+) -- (3);
%     \draw[line width=0.5mm] (2-) -- (w2);
%     \draw[line width=0.5mm] (3+) -- (w3);
%     \draw[line width=0.5mm] (3-) -- (w4);
%
% \end{tikzpicture}
% \end{adjustbox}
%
% \caption[Insert an abbreviated caption here to show in the List of Figures]
% {The Stern-Gerlach experiment as described by the von Neumann measurement scheme. Each measurement outcome corresponds to a term in the time-evolved state (Eq 4.4). Notice that the measurement interaction results in a branching structure, which is represented here visually with a tree graph. }
% \label{Figure:Measurement:labelthis}
% \end{figure}

% There is no one ``measurement problem'' of standard quantum mechanics,
%
% By discarding the projection postulate, no notion of an undefined ``interaction with a classical apparatus'' is required to describe how states evolve when their properties are recorded. The various paradoxes and interpretational issues associated with state collapse are entirely circumvented.
%

\section{Einselection vs. Inselection}
We could solve the preferred basis problem by including the environment rather than the apparatus. Such an approach is called superselection. However, the apparatus must exist by Newton's third law, and it solves the problem without reference to the environment. So, it seems reasonable that the configuration of the measurement fixes the basis (reflected in dynamics), as the basis we observe is a property of the apparatus. We will see that the role of the environment is to record the history TODO.
