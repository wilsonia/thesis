\chapter{Conclusion}
The Stern-Gerlach experiment has been modeled using the consistent histories approach to quantum mechanics. Our description relies on no fundamental assumptions about measurement, instead describing it as a unitary physical process. Implementing the von Neumann measurement scheme, the measurement process results in entanglement of electron spin, electron position, and environment degrees of freedom. The environment continuously records the state of the electron, establishing ``facts of the universe'' so that measurement objectively occurs outside of the observer's perception.

The measurement model fills in the gap left by discarding the projection postulate, while the consistent histories approach extends the Born rule to assign probabilities outside the narrow scope of standard measurement. Probabilities are now assigned to any set of exhaustive and mutually exclusive outcomes, rather than outcomes of interactions with a ``classical measurement apparatus'' only. This generalization allows cosmologists to study quantum systems in the early universe before the existence of such an apparatus.

By fleshing out the Stern-Gerlach experiment in the consistent histories approach, we have provided an introduction to the approach itself. Targeting students with a spins-first education in quantum mechanics, we presented a more detailed quantum analysis of the Stern-Gerlach experiment. By motivating and establishing the interaction between the electron and the environment, we have prepared the student for future study of reduced density matrices and environmental decoherence \cite{Schlosshauer}.

\bibliography{references}{}
\addcontentsline{toc}{chapter}{Bibliography}
\bibliographystyle{osu2}
